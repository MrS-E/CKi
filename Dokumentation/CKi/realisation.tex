\section{Allgemein}
\label{sec:RealAllgemein}
Die Realisation des Projekts CKI zeichnet sich durch die Implementierung eines modularen neuronalen Netzwerks aus, das für Aufgaben wie die Erkennung handschriftlicher Ziffern konzipiert wurde. Zum Einsatz kamen dabei Standard-C++-Technologien sowie eine CMake-basierte Projektstruktur für das Build-Management. Die Kernstruktur besteht aus Klassen für Neuron, Layer, und Network, die die Basis des Netzwerks bilden, ergänzt durch eine Util-Klasse für Hilfsfunktionen, wie Aktivierungsfunktionen und Datenverarbeitung.
Zusätzlich gibt es im Ordner „Test“, welche Unit-Tests mit den entsprechenden Attrappen für die Datensätze. Die Unit-Tests sind mit Google-Tests implementiert.
\\
Die klare Trennung zwischen den Komponenten des Netzwerks und die Nutzung von CMake unterstreichen einen modernen Ansatz in der Softwareentwicklung, der Flexibilität und Erweiterbarkeit des Projekts fördert. 

\section{Abhängigkeiten}
\label{sec:RealAbhängigkeiten}
Das Projekt CKI kommt mit nur wenigen Abhängigkeiten aus.
Die dies es dennoch gibt, sind umso wichtiger für das erfolgreiche Ausführen der Applikation.
\\
Abhängigkeiten:
\begin{enumerate}
	\item \textbf{googletest:} \\
	\textbf{Version (Tag):} v1.14.0 \\
	\textbf{Git:} \url{https://github.com/google/googletest} \\
	\textbf{Beschreibung:} GoogleTest ist ein C++-Framework für Unit-Tests, das Assertions für die Überprüfung von Code und Funktionen für die Organisation und Ausführung von Tests bietet.
	\item \textbf{nlohmann/json:} \\
	\textbf{Version (Tag):} v3.11.3 \\
	\textbf{Git:} \url{https://github.com/nlohmann/json} \\
	\textbf{Beschreibung:} Nlohmann/json ist eine moderne, header-only C++ Bibliothek für die Verarbeitung von JSON-Daten, die einfache Integration und intuitive Nutzung bietet.
	\item \textbf{wichtounet/mnist:} (nicht mehr in Verwendung)\\
	\textbf{Version (Commit):} 3b65c35 \\
	\textbf{Git:} \url{https://github.com/wichtounet/mnist} \\
	\textbf{Beschreibung:} Wichtounet/mnist ist ein einfacher C++-Reader für den MNIST-Datensatz, der es ermöglicht, Trainings- und Testbilder sowie Labels zu lesen und zu verwenden.
\end{enumerate}

\section{Architektur}
\label{sec:RealArchitektur}
\subsection{Klassen}
\label{sec:RealKlassen}
\subsubsection{Netzwerk}
\label{sec:RealNetzwerk}
Die Network-Klasse repräsentiert das neuronale Netzwerk. Es unterstützt die Initialisierung des Netzwerks mit einer bestimmten Anzahl von Eingabe- und Ausgabeneuronen sowie eine variable Anzahl von versteckten Schichten und deren Grössen. Die Klasse bietet Funktionen für das Training des Netzwerks mit gegebenen Eingaben und Zielwerten, die Überprüfung der Netzwerkleistung, Vorhersagen für neue Eingaben, die Durchführung von Vorwärts- und Rückwärtspropagation, sowie das Speichern und Laden des Netzwerks in bzw. aus einer Datei. 
\subsubsection{Layer}
\label{sec:RealLayer}
Die Layer-Klasse definiert einen Layer im neuronalen Netzwerk, bestehend aus mehreren Neuronen. Sie bietet Funktionen zum Berechnen der Ausgaben der Neuronen basierend auf Eingaben, Initialisieren und Setzen von Gewichten, Aktualisieren von Gewichten und Biases auf Basis von Fehlern und Lernrate, sowie zur Fehlerberechnung im Vergleich zu Zielwerten. Jedes Layer-Objekt enthält eine Liste von Neuron-Objekten, die die Neuronen in diesem Layer repräsentieren. 
\subsubsection{Neuron}
\label{sec:RealNeuron}
Die Neuron-Klasse stellt ein einzelnes Neuron dar, inklusive seiner Gewichte, Eingaben, Aktivierungsfunktion, Bias und Summe der gewichteten Eingaben. Sie bietet Funktionen zur Berechnung der Aktivierung basierend auf den Eingaben, der Ableitung der Aktivierungsfunktion, der Fehlerberechnung im Vergleich zu einem Zielwert, sowie zur Speicherung der Gewichte und des Biases. 
\subsubsection{Utility}
\label{sec:RealUtility}
Die Util-Klasse bietet statische Hilfsfunktionen für das neuronale Netzwerk, darunter die Sigmoid-Aktivierungsfunktion und ihre Ableitung sowie Funktionen zum Lesen von MNIST-Bilddaten und Labels aus Dateien. Diese Hilfsfunktionen sind essenziell für die Vorverarbeitung von Eingabedaten und die Implementierung der Lernmechanismen im Netzwerk. 
\subsection{Ordnerstruktur}
\label{sec:RealOrdnerstruktur}
Im Kern des Projekts stehen die Hauptdateien, welche die essenziellen Klassen wie Neuron, Layer, Network und Util enthalten. Diese sind grundlegend für die Funktionalität des neuronalen Netzwerks. Die CMakeLists.txt unterstützt das Build-Management, vereinfacht die Kompilierung und Konfiguration. Der .gitignore sorgt dafür, dass unnötige Dateien und Ordner nicht in die Versionskontrolle einfliessen. Ein speziell dafür vorgesehener Test-Ordner beinhaltet Tests zur Überprüfung der Funktionalität, was die Zuverlässigkeit des Systems sicherstellt. Zusätzlich runden Dummy-Files und ein eigener Ordner für die MNIST-Datasets das Projekt ab, indem sie die Datenhaltung für Training und Tests des Netzwerks erleichtern. 
\\
Das CKI-Projekt ist strukturiert in:
\begin{itemize}
	\item \textbf{Hauptdateien:} 
	Enthalten die Klassen Neuron, Layer, Network, und Util für die Kernlogik des neuronalen Netzwerks.
  \item \textbf{CMakeLists.txt:} 
	Für das Build-Management erleichtert das Kompilieren und die Konfiguration des Projekts.
  \item \textbf{.gitignore:} 
	Definiert Dateien und Ordner, die von Git-Versionierung ausgeschlossen sind, wie Build-Artefakte und IDE-spezifische Dateien.
  \item \textbf{Test-Ordner:} 
	Beinhaltet Testfälle zur Überprüfung der Funktionalität einzelner Komponenten und des Gesamtsystems.
	\item \textbf{Dummy-Files:} 
	Zusätzliche Dateien für Testzwecke.
	\item \textbf{MNIST-Datasets Ordner:} 
	Speichert die Datensätze für das Training und Testen des Netzwerks, insbesondere für die Erkennung handschriftlicher Ziffern.
\end{itemize}

\section{Code}
\subsection{Konzept}

\subsection{Algorithmen}
\subsubsection{ Forward-Propagation}
\label{sec:RealForwardPropagation}
Die Forward-Propagation ist ein grundlegender Prozess in neuronalen Netzwerken, der es ermöglicht, Vorhersagen auf Basis von Eingabedaten zu treffen. Dabei werden die Eingabedaten durch das Netzwerk von der Eingabeschicht über eine oder mehrere versteckte Schichten bis zur Ausgabeschicht vorwärts geleitet. Jede Schicht besteht aus Neuronen, die über Gewichte mit den Neuronen der vorherigen Schicht verbunden sind. Die Daten werden in jedem Neuron durch eine Summationsfunktion verarbeitet, die die gewichteten Eingaben aufsummiert und einen Bias-Wert hinzufügt. Das Ergebnis dieser Summation wird dann durch eine Aktivierungsfunktion geleitet, um die Ausgabe des Neurons zu bestimmen.
\\
Die Aktivierungsfunktion bestimmt, wie Neuronen ihre Eingaben in Ausgaben umwandeln und ist entscheidend für die Fähigkeit des Netzwerks, komplexe Muster in den Daten zu erkennen. Beliebte Aktivierungsfunktionen sind die Sigmoid-, Tanh- und ReLU-Funktion.
\\
Sobald die Eingabedaten durch das Netzwerk propagiert worden sind und die Ausgabeschicht erreicht haben, wird das Ergebnis mit dem tatsächlichen Wert verglichen, um den Fehler der Vorhersage zu bestimmen. Dieser Fehler wird dann in einem separaten Prozess, der als Backpropagation bekannt ist, verwendet, um die Gewichte im Netzwerk anzupassen und die Vorhersagegenauigkeit zu verbessern.\footnote{Nüesch 2023}
\subsubsection{Back-Propagation}
\label{sec:RealBackPropagation}
Die Backpropagation, kurz für „backward propagation of errors“, ist ein Schlüsselmechanismus im Training neuronaler Netzwerke. Dieser Algorithmus ermöglicht es, die Gewichte des Netzwerks so anzupassen, dass der Gesamtfehler bei der Vorhersage minimiert wird. Backpropagation wird nach der Forward-Propagation angewendet, nachdem eine Vorhersage durch das Netzwerk gemacht und der Fehler zwischen der Vorhersage und dem tatsächlichen Wert berechnet wurde.
\\
Der Prozess der Backpropagation besteht aus zwei Hauptphasen: der Berechnung des Gradienten des Fehlers bezüglich aller Gewichte im Netzwerk und der anschliessenden Anpassung dieser Gewichte in die Richtung, die den Fehler minimiert. Der Gradient gibt an, in welche Richtung die Gewichte verändert werden müssen, um den Fehler zu verringern, und die Grösse der Anpassung wird durch die Lernrate bestimmt.
\\
Backpropagation nutzt die Kettenregel der Differenzialrechnung, um die Fehlergradienten für die Gewichte jeder Schicht vom Ausgang zurück zum Eingang effizient zu berechnen. Der berechnete Fehlergradient für jede Gewichtung zeigt, wie eine kleine Änderung in diesem Gewicht den Gesamtfehler beeinflusst. Durch die systematische Anpassung der Gewichte basierend auf diesen Gradienten kann das Netzwerk schrittweise verbessert werden, um genauere Vorhersagen zu liefern.
\\
Insgesamt ermöglicht Backpropagation das effiziente Training tiefer neuronaler Netzwerke, indem es systematisch die Netzwerkgewichte anpasst, um den Fehler zwischen den Vorhersagen des Netzwerks und den tatsächlichen Werten zu minimieren, was zu einer verbesserten Modellleistung führt.\footnote{Nüesch 2023}


\subsection{Schlüsselpassagen & Snippets} %stolpersteine
